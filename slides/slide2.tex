\section{A história do Teorema da divergência}
\begin{frame}
    \frametitle{A história do teorema da divergência}
    \begin{itemize}
        \item \textbf{George Green (1793-1841)}:
              \begin{itemize}
                  \item Matemático autodidata
                  \item "Um ensaio sobre a aplicação da análise matemática às Teorias de Elasticidade e Magnetismo."
              \end{itemize}
        \item \textbf{Gauss (1777-1855)}:
              \begin{itemize}
                  \item Príncipe dos matemáticos
                  \item Teorema de Gauss
              \end{itemize}
        \item \textbf{Michal V. Ostrogradskij (1801-1861)}:
              \begin{itemize}
                  \item Declarou e provou a fórmula
              \end{itemize}
    \end{itemize}
\end{frame}

\begin{frame}
    \frametitle{A história do teorema de Stokes}
    \begin{itemize}
        \item \textbf{Stokes (1819-1903):}
              \begin{itemize}
                  \item Referência em dinâmica dos fluídos, óptica e física matemática
                  \item Utiliza o rotacional em seu teorema e fórmula
              \end{itemize}
    \end{itemize}
    \vspace{5mm}
    \begin{equation*}
        \oint_C F \cdot dr = \iint_S rot F dS
    \end{equation*}
\end{frame}

\section{Aplicações}
\begin{frame}
    \frametitle{Aplicações}
    \begin{itemize}
        \item Mecânica dos fluídos
        \item Eletricidade
        \item Magnetismo
    \end{itemize}
\end{frame}
